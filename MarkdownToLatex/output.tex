% Options for packages loaded elsewhere
\PassOptionsToPackage{unicode}{hyperref}
\PassOptionsToPackage{hyphens}{url}
\documentclass[
]{article}
\usepackage{xcolor}
\usepackage[margin=1in]{geometry}
\usepackage{amsmath,amssymb}
\setcounter{secnumdepth}{-\maxdimen} % remove section numbering
\usepackage{iftex}
\ifPDFTeX
  \usepackage[T1]{fontenc}
  \usepackage[utf8]{inputenc}
  \usepackage{textcomp} % provide euro and other symbols
\else % if luatex or xetex
  \usepackage{unicode-math} % this also loads fontspec
  \defaultfontfeatures{Scale=MatchLowercase}
  \defaultfontfeatures[\rmfamily]{Ligatures=TeX,Scale=1}
\fi
\usepackage{lmodern}
\ifPDFTeX\else
  % xetex/luatex font selection
\fi
% Use upquote if available, for straight quotes in verbatim environments
\IfFileExists{upquote.sty}{\usepackage{upquote}}{}
\IfFileExists{microtype.sty}{% use microtype if available
  \usepackage[]{microtype}
  \UseMicrotypeSet[protrusion]{basicmath} % disable protrusion for tt fonts
}{}
\makeatletter
\@ifundefined{KOMAClassName}{% if non-KOMA class
  \IfFileExists{parskip.sty}{%
    \usepackage{parskip}
  }{% else
    \setlength{\parindent}{0pt}
    \setlength{\parskip}{6pt plus 2pt minus 1pt}}
}{% if KOMA class
  \KOMAoptions{parskip=half}}
\makeatother
\setlength{\emergencystretch}{3em} % prevent overfull lines
\providecommand{\tightlist}{%
  \setlength{\itemsep}{0pt}\setlength{\parskip}{0pt}}
\usepackage{bookmark}
\IfFileExists{xurl.sty}{\usepackage{xurl}}{} % add URL line breaks if available
\urlstyle{same}
\hypersetup{
  hidelinks,
  pdfcreator={LaTeX via pandoc}}

\author{}
\date{}

\begin{document}

\section{Supervisory Guidance Document: Academic Review and Strategic
Advice for Thesis
Progression}\label{supervisory-guidance-document-academic-review-and-strategic-advice-for-thesis-progression}

\subsection{Introduction}\label{introduction}

This document serves as a comprehensive synthesis of the discussions,
feedback, and academic insights from the thesis supervision meeting
between Matt (Supervisor) and Mason (Student). Its purpose is to
encapsulate the depth and breadth of the intellectual engagement, to
clarify theoretical and methodological reservations, and to consolidate
the actionable advice provided. It goes beyond mere minutes,
articulating the underlying academic reasoning, expectations for thesis
structure and content, as well as practical steps for next-stage
development.

\begin{center}\rule{0.5\linewidth}{0.5pt}\end{center}

\subsection{1. Epiphany and Motivation in Research
Practice}\label{epiphany-and-motivation-in-research-practice}

A key theme opened the discussion: the process leading to a
research-related ``epiphany'' and how a systematic review can provide
structure, motivation, and ultimately, clarity. The student's experience
demonstrates that moments of insight often follow sustained effort. This
underscores an important lesson for academic work: breakthroughs are not
serendipitous, but arise from persistent engagement with literature and
analytical frameworks.

\begin{itemize}
\tightlist
\item
  \textbf{Take-home Message:} The process of conducting a thorough
  literature review, even one that feels diffuse at the outset, is
  critical to structuring one's own thinking and arriving at research
  clarity. Motivation waxes and wanes, but consistent ``work in'' is
  what leads to insight and robust research direction.
\end{itemize}

\begin{center}\rule{0.5\linewidth}{0.5pt}\end{center}

\subsection{2. Structural Critique of the Research
Proposal}\label{structural-critique-of-the-research-proposal}

\subsubsection{2.1. High-Level Rationale vs.~Methodological
Specificity}\label{high-level-rationale-vs.-methodological-specificity}

The draft under review was found to possess a strong high-level
rationale and theoretical framing, but was significantly lacking
methodological specificity.

\begin{itemize}
\tightlist
\item
  The document needed to pass the ``handover test'': can an RA or
  programmer implement the experiment based strictly on the research
  proposal? At present, the draft would be insufficient for independent
  execution.
\item
  While certain proposals lack theoretical rigor, and others lack
  operational detail, the expectation is for both to be present and
  clearly articulated.
\end{itemize}

\paragraph{\texorpdfstring{\textbf{Key
Advice:}}{Key Advice:}}\label{key-advice}

\begin{itemize}
\tightlist
\item
  Articulate, with specificity, experimental methods:

  \begin{itemize}
  \tightlist
  \item
    Who are the participants?
  \item
    What are the stimuli (types, numbers, characteristics)?
  \item
    What is the procedure (step-wise timeline, stimuli exposure,
    duration)?
  \item
    What are the hypotheses and corresponding predictions?
  \end{itemize}
\end{itemize}

\subsubsection{2.2. Linking Theory to Method and
Prediction}\label{linking-theory-to-method-and-prediction}

A critical shortcoming identified was the insufficiently explicit
linkage between theory, experimental manipulation, and predicted
outcomes. Logical steps that underpin the rationale should be
``painfully obvious,'' i.e., spelled out in full sentences that
sequentially build the argument.

\paragraph{\texorpdfstring{\textbf{Key
Advice:}}{Key Advice:}}\label{key-advice-1}

\begin{itemize}
\tightlist
\item
  Avoid compressing complicated logical steps into brief or ambiguous
  sentences.
\item
  Differentiate clearly between related but distinct theoretical
  constructs (e.g., internal clock vs.~attentional gate models).
\item
  Systematically walk from theory, through hypothesis, to experimental
  procedure and prediction.
\end{itemize}

\begin{center}\rule{0.5\linewidth}{0.5pt}\end{center}

\subsection{3. Theoretical Framework: Clock Models and Attentional
Gate}\label{theoretical-framework-clock-models-and-attentional-gate}

\subsubsection{3.1. Conceptual
Distinctions}\label{conceptual-distinctions}

Theories of time perception such as internal clock models and
attentional gate models were referenced, yet their differences and their
roles in the experiment must be spelled out more clearly. The supervisor
advised reverse engineering: start from the predictions of each model
within the specific context of your experiment, then build the rationale
back to the theory.

\begin{itemize}
\tightlist
\item
  \textbf{Internal Clock Model:} Perceived duration is a function of
  accumulated internal pulses.
\item
  \textbf{Attentional Gate Model:} Allocation of attention modulates
  pulse accumulation; diverting attention reduces perceived passage of
  time.
\end{itemize}

\paragraph{\texorpdfstring{\textbf{Key
Advice:}}{Key Advice:}}\label{key-advice-2}

\begin{itemize}
\tightlist
\item
  For each theory, specify what result it predicts in your paradigm.
\item
  Explicitly map each experimental manipulation onto the constructs
  involved in each theory.
\item
  Clarify whether the design allows discrimination between competing
  theories, or simply tests a shared prediction.
\end{itemize}

\begin{center}\rule{0.5\linewidth}{0.5pt}\end{center}

\subsection{4. Conceptual Clarity: Cognitive Load, Perceptual Fluency,
and Visual
Categorization}\label{conceptual-clarity-cognitive-load-perceptual-fluency-and-visual-categorization}

Several terms (cognitive resources, perceptual fluency, categorization
difficulty) were raised but ambiguously defined.

\paragraph{\texorpdfstring{\textbf{Key
Advice:}}{Key Advice:}}\label{key-advice-3}

\begin{itemize}
\tightlist
\item
  Clearly define specialized terms upon first use.
\item
  Anchor claims in empirical evidence (i.e., ``Smith et al.~(Year)
  demonstrated X\ldots''), not just logical inference.
\item
  Be literal; avoid poetic language (``challenge ability to
  categorize''), and instead use physicist-like precision (``reducing
  resolution decreases categorization accuracy, as measured
  by\ldots{}'').
\item
  Build the conceptual logic \emph{step by step}, justifying every link.
\end{itemize}

\begin{center}\rule{0.5\linewidth}{0.5pt}\end{center}

\subsection{5. Hypothesis Formation and Statistical
Analysis}\label{hypothesis-formation-and-statistical-analysis}

\subsubsection{5.1. Explicit Hypotheses}\label{explicit-hypotheses}

Hypothesis statements were identified as present but not explicit. The
supervisor demanded clear, unambiguous hypotheses with stated
predictions and, if possible, predicted effect sizes.

\paragraph{\texorpdfstring{\textbf{Key
Advice:}}{Key Advice:}}\label{key-advice-4}

\begin{itemize}
\tightlist
\item
  State hypotheses in ``if\ldots then\ldots'' format, directly
  reflecting the manipulation and outcome measures.
\item
  Denote expected directionality of effects (e.g., ``Lower resolution
  images will yield shorter perceived durations compared to high
  resolution images'').
\item
  Consider, at least qualitatively, the size of expected effects (``We
  anticipate a moderate effect in line with Smith et al., 2020'').
\end{itemize}

\subsubsection{5.2. Statistical Planning}\label{statistical-planning}

Analysis strategy was insufficiently mapped to hypotheses. The need for
clarity in planned statistical comparisons was heavily stressed.

\begin{itemize}
\tightlist
\item
  Know your design: e.g., 3 (resolution: high, medium, low) x 2
  (familiarity: old vs.~new) mixed/within/between?
\item
  Detail what each ANOVA main effect and interaction means for your
  hypotheses.
\item
  Pre-plan post-hoc or planned comparisons. Don't just `run an
  ANOVA'---know which pairwise or contrast comparisons address which
  hypotheses.
\end{itemize}

\paragraph{\texorpdfstring{\textbf{Key
Advice:}}{Key Advice:}}\label{key-advice-5}

\begin{itemize}
\tightlist
\item
  Specify analysis plan in line with the design.
\item
  Know what \emph{significant main effects} and \emph{interactions} will
  and will not tell you.
\item
  Be prepared to justify every statistical test with reference to your
  theoretical predictions.
\end{itemize}

\begin{center}\rule{0.5\linewidth}{0.5pt}\end{center}

\subsection{6. Research Question: Clarification and
Framing}\label{research-question-clarification-and-framing}

The research question was discussed at length. The supervisor emphasized
that Mason's work should focus not on the manipulation per se (e.g.,
``What is the effect of face resolution on time perception?''), but on
the psychological process being probed by manipulating resolution.

\paragraph{\texorpdfstring{\textbf{Key
Advice:}}{Key Advice:}}\label{key-advice-6}

\begin{itemize}
\tightlist
\item
  Frame the research question at the process/mechanism level (``How does
  interfering with perceptual fluency via resolution reduction affect
  temporal estimation?'').
\item
  Use experimental manipulations (image resolution, familiarity,
  orientation) as \emph{proxies} for underlying cognitive processes
  (perceptual expertise, fluency, attention allocation).
\end{itemize}

\begin{center}\rule{0.5\linewidth}{0.5pt}\end{center}

\subsection{7. Confounds, Interpretation, and Design
Limitations}\label{confounds-interpretation-and-design-limitations}

A sophisticated point was raised regarding confounds: manipulating image
resolution not only changes categorization difficulty or familiarity,
but also basic visual properties (contrast, brightness, spatial
frequency), which themselves can affect time perception.

\paragraph{\texorpdfstring{\textbf{Key
Advice:}}{Key Advice:}}\label{key-advice-7}

\begin{itemize}
\tightlist
\item
  Preemptively acknowledge interpretive ambiguities (e.g., is it lower
  familiarity, disrupted perceptual fluency, or low-level visual
  features driving the observed effects?).
\item
  In both proposal and later discussion, cite literature demonstrating
  effects of such low-level features on time perception.
\item
  Present these not merely as limitations but as inherent complexities
  in perceptual manipulation.
\end{itemize}

\begin{center}\rule{0.5\linewidth}{0.5pt}\end{center}

\subsection{8. Writing Style and
Communication}\label{writing-style-and-communication}

Finally, the supervisor urged a clear, literal, and non-redundant style,
emphasizing that thoroughness and explicit stepwise argumentation trump
brevity or literary flourish in academic writing.

\paragraph{\texorpdfstring{\textbf{Key
Advice:}}{Key Advice:}}\label{key-advice-8}

\begin{itemize}
\tightlist
\item
  Err on the side of repetition when articulating the logical chain from
  theory to prediction.
\item
  ``Hit the reader over the head'' with your main points; crucial logic
  and predictions can be restated for clarity across introduction,
  hypothesis, method, and analytic sections.
\item
  Clarity and precision are paramount.
\end{itemize}

\begin{center}\rule{0.5\linewidth}{0.5pt}\end{center}

\subsection{9. Next Steps: Actionable
Tasks}\label{next-steps-actionable-tasks}

Based on the feedback and the overarching goals discussed, the following
detailed action plan emerges:

\subsubsection{Immediate Writing Tasks}\label{immediate-writing-tasks}

\begin{itemize}
\tightlist
\item
  \textbf{Expand theoretical rationale:} Develop each logical step in
  detail, defining all variables and concepts.
\item
  \textbf{Draft clear hypotheses:} Use explicit, directional statements.
\item
  \textbf{Detail methodological plan:} Specify every relevant detail a
  collaborator or programmer would need.
\item
  \textbf{Explicit statistical analysis plan:} Define exactly which
  tests address which hypothesis.
\end{itemize}

\subsubsection{Conceptual/Analytical
Tasks}\label{conceptualanalytical-tasks}

\begin{itemize}
\tightlist
\item
  \textbf{Generate predicted outcomes:} For each theoretical model,
  diagram or list the specific expected results.
\item
  \textbf{Review relevant theory:} Further differentiate internal clock
  from attentional gate, and fluency from cognitive load.
\item
  \textbf{Preempt confounds:} Note visual-perceptual changes introduced
  by your manipulations.
\end{itemize}

\subsubsection{Researcher Development}\label{researcher-development}

\begin{itemize}
\tightlist
\item
  \textbf{Consult statistics resources:} Review factorial ANOVA, main
  effects, interactions, post-hoc and planned contrasts, using both
  textbooks and accessible tutorials.
\item
  \textbf{Reflect on research question:} Articulate and refine your
  process/mechanism-focused research aim.
\end{itemize}

\begin{center}\rule{0.5\linewidth}{0.5pt}\end{center}

\subsection{10. Summary: Essence and Key
Takeaways}\label{summary-essence-and-key-takeaways}

This meeting foregrounded the vital importance of methodical clarity,
conceptual precision, and structured argument in psychological research
proposals. Mason was encouraged to: - Write with literal clarity, -
Explicitly connect theory to hypothesis to method to analysis, - Predict
and plan for both the findings and their possible confounds, - Approach
experimental manipulation as a means to probe underlying mechanisms, not
ends in themselves.

Above all, the meeting reinforced the principle that insight and clarity
arise from deep and sustained engagement with both the literature and
the logical structure of one's own research.

\end{document}
